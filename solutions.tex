%CS-113 S18 HW-5
%Released: 3-April-2018
%Authors: Abdullah Zafar.


\documentclass[addpoints]{exam}

% Header and footer.
\pagestyle{headandfoot}
\runningheadrule
\runningfootrule
\runningheader{CS 113 Discrete Mathematics}{Homework V}{Spring 2018}
\runningfooter{}{Page \thepage\ of \numpages}{}
\firstpageheader{}{}{}

\boxedpoints
\printanswers
\usepackage[table]{xcolor}
\usepackage{amsfonts,graphicx,amsmath,hyperref,amssymb}
\hypersetup{
    colorlinks=true,
    linkcolor=blue,
    urlcolor=cyan,
}

\title{Habib University\\CS-113 Discrete Mathematics\\Spring 2018\\HW 5 Solutions}
\date{Released: 3rd April, 2018}


\begin{document}
\maketitle

\begin{questions}



\question
Prove, by double counting:
\begin{parts}
  \part $\Sigma^{n}_{k=1} k \binom{n}{k} = n2^{n-1}$ 
  \begin{solution}
  
  	R.H.S.: The number of pairs ($x,subset$)  where $x \in [n]$ and $subset \subseteq \{[n] \backslash x\}$ is  $n2^{n-1}$.
  	
  	L.H.S.: Alternatively, to create pairs, first choose a subset $A$ of size $k$ from $n$, and then create pairs ($x,subset$), where $x \in A$ and $subset = A \backslash x$. Since the first step can be done in $\binom{n}{k}$ ways, the pairs can be created in $k\binom{n}{k}$ ways, where $k = 1,2,...,n$. In total, there are $\Sigma^{n}_{k=1}k \binom{n}{k}$ pairs.
  \end{solution}

  \part $\Sigma^n_{k=-m} \binom{m+k}{r} \binom{n-k}{s} = \binom{m+n+1}{r+s+1}$
  \begin{solution}
    
    R.H.S.: The number of ways to choose subsets of size $r + 1 + s$ from $[m + n +1]$.
    
    L.H.S.: Alternatively, let $A$ be a subset of size $r + 1 + s$ chosen from $[m + n +1]$ and let $x=1,2,...,(m+n+1)$, be the $r+1^{th}$ element in ordered $A$. Then the number of such subsets is \[\Sigma^{n+m}_{k=0}\binom{k}{r} 1 \binom{n+m-k}{s}\]  
    where we choose $r$ objects from the first $k$, and then $s$ objects from the last $n + m - k$. The 1 in the middle of the summation is symbolic of the fixed $r+1^{th}$ element. To complete the proof, we observe that \[\Sigma^{n+m}_{k=0}\binom{k}{r} 1 \binom{n+m-k}{s}= \Sigma^{n}_{k=-m}\binom{m+k}{r}\binom{n-k}{s}  \] 
  \end{solution}
\end{parts}

\question  You and a friend play a game with a pile of $N$ gold coins. On each turn, a player can remove 1, 3, or 6 coins from the pile. The winner is the one who takes the last coin. For $0 < N < 1000$, how many starting positions are winning positions for the first player?


  \begin{solution}
    666  (I will add the complete derivation soon).
  \end{solution}
  
\question 
Six friends, who each bought a gift, distribute them amongst each other such that everyone receives a gift. How many ways are there to distribute the gifts such that no friend gets their own gift?



  \begin{solution}
   265 (I will add the complete derivation soon).
  \end{solution}

\question [0]
Find the number of non-negative integer solutions of the equation

\[x_1 + x_2 + \cdots + x_n = y \;\;\;\;\;\;\text{where } x_1 < x_2 < \cdots < x_n\]

\question 
Your friend's been working hard on developing the covert \textit{Kabootar Messaging System (KMS)}. Their technique transforms each bit string into a unique bit string before sending over an insecure network. Once on the other side, the original bit string is fully reconstructed. Your friend claims that, on average, the \textit{KMS} decreases the length of strings before transmission, and in any case, does not increase the length of a string. Prove that they are wrong.

 \begin{solution}
	For sake of absurdity, suppose that $KMS$ does what it promises, i.e., generates a bijection $A \rightarrow B$, where $A$ is the set of all strings over some alphabet $\Sigma$ and $B$ the set of encrypted strings over $\Sigma$, such that on average, strings in $A$ are mapped to smaller strings in $B$ and none to larger strings. Let $A_n \subset A$ be the first $n$ strings in lexicographic order in $A$. Similarly let $B_n \subset B$ be the first $n$ strings in lexicographic order in $B$. According to $KMS$, there exists a bijection from $A_n \rightarrow B_n$.  But since the average length of strings in $B_n$ is smaller than $A_n$, and no string in $B_n$ is larger (in lexicographic terms) than any string in $A_n$, $B_n$ must contain less strings than $A_n$. Therefore, $|B_n| < n$, and the original claim is absurd.   
\end{solution}

\end{questions}

\end{document}