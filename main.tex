%CS-113 S18 HW-5
%Released: 16-March-2018
%Deadline: 30-March-2018 7.00 pm
%Authors: Abdullah Zafar, Waqar Saleem.


\documentclass[addpoints]{exam}

% Header and footer.
\pagestyle{headandfoot}
\runningheadrule
\runningfootrule
\runningheader{CS 113 Discrete Mathematics}{Homework V}{Spring 2018}
\runningfooter{}{Page \thepage\ of \numpages}{}
\firstpageheader{}{}{}

\boxedpoints
\printanswers
\usepackage[table]{xcolor}
\usepackage{amsfonts,graphicx,amsmath,hyperref,amssymb}

\title{Habib University\\CS-113 Discrete Mathematics\\Spring 2018\\HW 5}
\author{$<your ID>$}  % replace with your ID, e.g. oy02945
\date{Due: 19h, 30th March, 2018}


\begin{document}
\maketitle

\begin{questions}



\question
Prove, by double counting:
\begin{parts}
  \part $\Sigma^{n}_{i=1} k \binom{n}{k} = n2^{n-1}$ 
  \begin{solution}
    % Write your solution here
  \end{solution}

  \part $\Sigma^n_{k=-m} \binom{m+k}{r} \binom{n-k}{s} = \binom{m+n+1}{r+s+1}$
  \begin{solution}
    % Write your solution here
  \end{solution}
\end{parts}

\question  You and a friend play a game with a pile of $N$ gold coins. On each turn, a player can remove 1, 3, or 6 coins from the pile. The winner is the one who takes the last coin. For $0 < N < 1000$, how many starting positions are winning positions for the first player?


  \begin{solution}
    % Write your solution here
  \end{solution}
  
\question 
Six friends, who each bought a gift, distribute them amongst each other such that everyone receives a gift. How many ways are there to distribute the gifts such that no friend gets their own gift?



  \begin{solution}
    % Write your solution here
  \end{solution}

\question
Find the number of non-negative integer solutions of the equation

\[x_1 + x_2 + \cdots + x_n = y \;\;\;\;\;\;\text{where } x_1 < x_2 < \cdots < x_n\]

    
   \begin{solution}
    % Write your solution here
  \end{solution}

\question 
Your friend's been working hard on developing the covert \textit{Kabootar Messaging System (KMS)}. Their technique transforms each bit string into a unique bit string before sending over an insecure network. Once on the other side, the original bit string is fully reconstructed. Your friend claims that, on average, the \textit{KMS} decreases the length of strings before transmission, and in any case, does not increase the length of a string. Prove that they are wrong.

\end{questions}

\end{document}
